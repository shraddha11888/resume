\documentclass[margin]{res} 
% the margin option causes section titles to appear to the left of body text 
\usepackage{verbatim}

\sectionwidth=0.9in
\linespread{1.15}

%\usepackage{helvetica} % uses helvetica postscript font (download helvetica.sty)
%\usepackage{newcent}   % uses new century schoolbook postscript font 

\begin{document}

\vspace*{-0.5in}

\name{Shraddha Sridhar\\[8pt]} % the \\[12pt] adds a blank line after name

\address{ \textit{email}: ssridh14@asu.edu\\ \textit{phone}:  +1 (480) 327-7224}

\begin{resume} 

\section{Objective}
To consolidate and effectively apply my skill set to real world challenges in Electronic Circuit Design

\section{Education}
{\bf M.S. in Electrical Engineering, Arizona State University} \hfill  May 2012\\
GPA: 3.55 / 4\\
\textit{Courses}: VLSI Design, Advanced VLSI Design, Advanced Analog Integrated Circuits, MEMS, Digital Systems and Circuits, Semiconductor Device Theory, Design of Engg. Experiments\\
\textit{Future Courses}: Advanced Hardware Systems Design, Low power Bioelectronics, Analog to Digital Converters

\vspace{-0.2in}
{\bf Bachelor of Engineering, Electronics and Communications}  \hfill June 2010\\
First class (66\%), Mumbai University, India\\
\textit{Courses}: Computer Architecture, Microcontrollers and Embedded Programming, Digital Signal Processing, Elements of Microprocessors, Electronic Instrumentation, Digital Communication, Microwave Devices and Circuits, Computer Communication Networks

\section{Experience}
{\bf Research Aide} \hfill May 2011 - present\\
{\bf Prof. Lawrence Clark, ASU}
\begin{itemize}
    \item Microprocessor Cache Design: Currently designing and performing timing analysis on a Radiation Hardened Microprocessor 16kB, 4-way set associative write through cache for a 45 nm SOI process. Created a fully customized layout for the SRAM subbank used in the cache, including the decoder and control circuitry.
    \item Functional validation of MIPS-4k based processor: Design of an automated validation environment for the functional validation of a Radiation Hardened MIPS-4k based processor. Automated the complete flow from generating tests using the scoreboard method till comparing the outputs using Perl scripts.The automated environment can create tests with the specified number, type and probability of pseudo-random instructions and can provide the result of validation. Also implemented directed tests for corner cases.  
    
\end{itemize} 

{\bf Research Aide} \hfill Jan 2011 - May 2011\\
{\bf Prof. Michael Goryll, ASU}
\begin{itemize}
    \item Implemented a switched capacitive transimpedance amplifier on an AMI process for measuring currents through single ion channels of cells, recorded using the patch clamp technique. The circuit will soon be sent for tape-out.
\end{itemize}
 
{\bf Project Trainee} \hfill July 2009 - May 2010\\
{\bf Bhabha Atomic Research Center (BARC), India}
\begin{itemize} 
    \item Designed a Digital Comb FIR filter and implemented it on a custom designed FPGA system for low noise application. Simulations were done on MATLAB to condition transducer signals at 25 Hz having all odd harmonics till 5KHz. SNR was improved by a factor of 3. Novel techniques include the use of 2048 coefficients of 32 bit precision for band pass filter using Kaiser window and DMA for the external RAM memory of the FPGA.
\end{itemize} 


\section{Teaching}
Teaching Associate for Advanced VLSI Design (EEE 625, ASU). Involves managing simulation labs on digital circuit design projects like cache design, register file array design.
\section{Publications}
S. Sridhar, J. Kapadnis, V.P. Mapare, S. Sonare, ``Real time locating Systems using RFID-Radar'', Electronics and Telecommunication Department, V.I.T, Mumbai, National Conference on Innovations in Electronics and Information Technology, 2009.

\section{Projects}
\begin{itemize}
\item
{\bf Digital}:  Implemented 8-bit FIR filter including layout and optimized the design for minimum Energy Delay Product. Implemented pipelining and timing concepts like time borrowing using latches and implemented carry select adder to minimize the delay. Also designed 4 bit digital comparator for high speed on .25um technology.
\item
{\bf Analog}: Designed and optimized 50 ohm driver amplifier, folded cascode opamp and different current mirrors and beta multiplier circuits. Optimized CMOS switches for low noise. Also designed a differentially sensed voltage controlled oscillator circuit.
\item
{\bf MEMS}: Implemented a MEMS based capacitive microphone array for improved directionality and sensitivity. Created a complete process flow for the design.
\end{itemize}

\section{Extracurricular \\ Activities}
\begin{itemize}
\item
{\bf Technical}: Head of the Technical Committee, V.I.T, 2008-2009. Organized Fervor, 2009, a state-wide technical symposium featuring various robotics events, paper presentations, technical quizzes and miscellaneous events. Also participated in Nexus 2008, a national level robotics event organized by IIT Bombay. Designed a robot that can pick boxes and place them on a raised platform.
\item
{\bf Literary}: Head of Magazine Committee, Electronicsand Telecommunication Student Association, V.I.T. Brought out the first department 
magazine covering the recent developments in Telecommunication. Passed several Sanskrit literature examinations equivalent to the degree Bachelor of Arts with distinction.
\item
{\bf Music} : Head of Music Committee, V.I.T, core member of Cultural Committee, V.I.T. Organized Verve 2009, the popular annual cultural festival of V.I.T. Also, offered various concerts in Indian Classical Music in Mumbai,India. Passed several music examinations with distinction.
\item
{\bf Sports}: Won throw-ball, kho-kho, khabaddi events as part of the college girls team.
\end{itemize}
% Tabulate Computer Skills; p{3in} defines paragraph 3 inches wide
\section{Computer \\ Skills}
   \begin{tabular}{l p{3.5in}}
        {\bf Software} &  Cadence IC, Diva Tools, HSpice, Spectre, Calibre, Nanotime, PrimeTime, Xilinx ISE, COMSOL, LEdit, Matlab, Eagle \\\\
    {\bf Languages} & C, C++, Perl, VHDL, Verilog, bash
 \end{tabular}


\end{resume} 
\end{document} 



